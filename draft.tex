\documentclass[11 pt , letterpaper , twoside , openright]{book}
\usepackage[utf8]{inputenc}

\usepackage[english]{babel}     	
\usepackage{graphicx}                   % om afbeeldingen in te kunnen laden   
\usepackage{tikz,tikz-3dplot}                       % om afbeeldingen te kunnen maken
\usetikzlibrary{calc}                   % meer opties in tikz (oa coordinaten bepalen)

\usepackage{wrapfig}              % meer controle over figuren
\usepackage{floatrow}
\usepackage[format=hang]{caption}
\usepackage{listings}                   % voor mooie code
\usepackage{amsmath, amssymb, amsfonts} % om wiskunde te typen
\usepackage{scalerel}                   % om symbolen te schalen
\let\proof\relax
\let\endproof\relax
\usepackage{amsthm}
\usepackage{bm}                         % Wiskundesymbolen vet met commando \bm{}
\usepackage{libertine}                  % lettertype
\usepackage{inconsolata}                % lettertype code
\usepackage[T1]{fontenc}                % betere lettertype encodering voor talen met accenten
\usepackage{setspace}
\usepackage{enumitem}
\usepackage{units}
\usepackage{multicol}
\usepackage{appendix}
\usepackage{titlesec}
\usepackage{algorithm}
\usepackage{physics}
\usepackage[noend]{algpseudocode}
\usepackage{tikz-feynman}
\usepackage{subcaption}
\usepackage{xfrac}
\tikzfeynmanset{compat=1.0.0}
\floatname{algorithm}{Algoritme}

\begin{document}

\chapter{Introduction}

The past decade social media has become ever more important in our daily lives. Nowadays most of us have, beside a real world life, a virtual life on social media. However, this increased engagement on social media may have a huge impact on the formation of opinions, not only on the individual scale, but also on a broader, societal, scale. It is known that individuals not only form their opinions through self-reflection, but also through interactions with other people and with their surroundings. The broad range of interactions people undergo on social media with people from all over the world can thus not be underestimated in the process of opinion formation. The understanding of the role of social media on the emergence of polarization and extremism in society is of uttermost importance.\\
 It is also known that some people change their opinions more easily than others, that is once in a while you encounter somebody that is really stubborn and persistent towards its own opinion. The aim of this thesis is to design toy models of opinion dynamics with stubborn actors on theoretical and real-world social networks to analyze the interplay between individual resistance to change and the network structure in the evolution of two competing opinions.\\
Opinion dynamics models have two important layers. On one hand, we introduce social networks to describe the underlying structure of social interactions, while on the other hand we need appropriate opinion dynamics/formation models to reproduce the opinion dynamics and opinion formation processes found in real life. 

\section{Complex networks}

Complex networks have become more and more popular for analyzing complex, dynamical systems. They are used in fields such as physics, economics, social sciences, biology, etc \cite{Costa2008}. These different fields are very diverge, but they have at least one common ground: they often deal with a large number of variables/components that interact with each other. In other words, in all these fields one encounters complex systems, systems where it is not possible to predict collective behavior based on the properties of the individual components alone \cite{Mata2020}. Complex systems often display phenomena such as non-linearity, emergence, spontanuous order,... %(maybe check cursus compl en crit?)
One of the tools to deal with these complex systems and to give us more insight in the possible underlying structures are complex networks. A network, also often called a graph, is a structure composed of nodes or vertices and a set of links (edges) that indicate the interactions between the nodes \cite{Mata2020}. Representing/modeling a complex system as a graph makes the system appear more simple and tractable, while it still includes the non-linearity of these systems. One can find the language to describe networks in mathematical graph theory. However, complex systems in real life situations often deal with a huge number of components, so the use of statistical and high-performance computing tools is inevitable \cite{Mata2020}.\\
The advantage of working with network models is that they reduce the level of complexity encountered in the real world, so that one can treat these systems in a more practical way. However we do want our models to display properties similar to the ones seen in real systems \cite{Mata2020}. Since many real systems are not static but evolve in time, this means that we don't only need to deal with static networks but also with temporal networks. Static networks have been widely studied and are often convenient for their analytic tractability, whereas temporal networks are, in some cases, more realistic \cite{Mata2020}. Some other important concepts in network theory are the degree distribution, clustering/community structure, connectivity, etc. These concepts will be explained in depth in section \ref{sec2}, but let us already anticipate a bit on the case of social networks. It is found that many real life social systems have a power law degree distribution. This heterogeneous or scale free degree distribution represents one of the three general properties of social networks. The other two are short distances, also referred to as small-world phenomenon, and high clustering \cite{Muchnik2013}. Ideally, our theoretical network should exhibit these three properties. Some theoretical networks, that are thoroughly studied, do not possess all of these three properties. However, they might still be used, because of their simplicity and ability to reproduce analytic results. When using these models one must always bear in mind their limitations to reproduce some properties encountered in real social systems. \\
One of the goals of this thesis is to investigate/determine the role of the underlying network structure on the formation and dynamics of opinions in the system. Some examples of theoretical network models used in this thesis are the Erd\H{o}s-R\'{e}nyi network, a clustered random network, the Watts-Strogatz model, etc. These will be explained thoroughly in section \ref{sec2}. %references to chapter 2 need to be adapted
% to do: read links 19, 21, 22, 23, 24 of https://arxiv.org/pdf/0711.3199.pdf

% maybe also talk about degree distributions (poisson, power law), power law behavior of many real systems, clustering + maybe talk about networks that will be used in this thesis?

\section{Opinion dynamics}

As stated before, the formation of an individual's opinion is the result of the interplay of many factors such as self-reflection, peer pressure, the individual's personality (eg. stubbornness), the information someone is exposed to, etc.  Opinion dynamics models should try to capture these complex processes in a simplified way. Several models have been developed ranging from simple binary models to more complex, continuous approaches \cite{Sirbu2016}. The basic idea of all opinion dynamic models is that the nodes or agents in a social network have a variable that represents their opinion and that is updated according to some predefined rules. These models are obviously a simplification of real world opinion dynamics. It is however shown that they do display a lot of aspects of real opinion formation such as agreement, transitions between order (consensus) en disorder (fragmentation), polarization, formation of echo chambers (clusters of people with the same opinion), etc \cite{Sirbu2016}. \\
The opinion of the agents in the model can be either discrete or continuous. This thesis deals with models where each agent can have one of the two opinions A or B and thus only deals with the case of discrete opinions. This might come across as a huge simplification of the real world complexity of opinion formation, but also in real life people often have to choose between two competing opinions (eg. republicans or democrats, being left-minded or right-minded, cat or dog, renting or buying a house,...). \\

%The rules that determine how an agent updates his or her opinion can include many different aspect, such as the higher or lower influence somebody can have on others, the resistance to change to a new opinion, peer pressure,... The structure of social networks plays an important role in aspects as peer pressure, whereas the algorithmic personalization used on social media platforms may interfere with the information someone is exposed to. Algorithmic personalization is the individualized way in which social media platforms show posts/information to each of its users.

% not finished, describe different models
% talk about updating rules used in thesis + concepts as resistance etc. 
% biblio: https://link.springer.com/article/10.1007/s13538-020-00772-9

\section{Complexity and social physics}

Another important tool in the study of opinion dynamics/formations in large groups of people is the use of statistical physics. This branch of physics offers a framework that relates microscopic properties of atoms, molecules,... to macroscopic observed behavior and has a wide field of applications inside and outside the world of physics. The observation that large number of people display collective, 'macroscopic' behavior begged for the use of the concepts and insights developed in statistical physics \cite{Sirbu2016}. The application of the theory of statistical physics on social phenomena is referred to as social physics or sociophysics. The 'microscopic' constituents are now individual humans who interact with a limited number of other individuals and in that way form complex, 'macroscopic' groups such as human societies. These 'macroscopic' groups display stunning regularities, transitions from disorder to order, the emergence of consensus, universality, etc. The statistical physics approach of these social systems tries to explain the found regularities at large scale as collective effects of the interactions among these individuals \cite{Sirbu2016}.
% talk about history


% talk somewhere about rise of social physics, the need to use statistical physics (because of macroscopic phenomena as scale invariance, emergence, polarization,...)
\chapter{'section2'}
\label{sec2}



\newpage
% bibliography
\bibliographystyle{abbrv} % can change this style
\bibliography{references}
% to run this go to correct location in terminal --> type: pdflatex name, (file name.tex) --> type: bibtex name --> type: pdflatex name --> type: pdflatex name 
\end{document}